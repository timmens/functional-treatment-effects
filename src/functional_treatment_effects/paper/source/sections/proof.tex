\section{Theory and Proofs}


\begin{definition}
    A mean-zero process $X = \{X(t) : t \in [0, 1]\}$ is said to be $f$-H\"older
    continuous if
    \[
        \mathbb{E}\left[(X(s) - X(t))^2 \right] \leq f(|s - t|)
    \]
    for small $|s-t|$.
\end{definition}


\begin{definition}
    We say a function $f: [0, 1] \to \mathbb{R}$ satisfies the integral condition A
    (ICA) if
    \[
        \int_0^\infty y^{-3/2} \sqrt{f(y)} \, dy < \infty.
    \]
\end{definition}

\begin{definition}
    If there exists a function $f$ that is non-negative and non-decreasing in a
    neighborhood around 0, such that $X$ is $f$-H\"older continuous and $f$ satisfies
    the ICA, then we say $X$ satisfies the ICA.
\end{definition}


\begin{proposition}
    Let $X$ be a mean-zero process with stationary kernel, i.e. there is a function
    $\tilde{c} : [0, 1] \to \mathbb{R}$ such that $\mathbb{E}[X(s) X(t)] = \tilde{c}(|s
    - t|)$ for all $s, t$. Then $X$ satisfies the ICA if and only if $2 (\tilde{c}(0) -
    \tilde{c})$ satisfies the ICA.
\end{proposition}
\begin{proof}
    Notice that
    \begin{align}
        \mathbb{E}\left[(X(s) - X(t))^2\right]
        &= \mathbb{E}\left[X(s)^2 + X(t)^2 - 2 X(s) X(t)\right] \\
        &= \mathbb{E}[X(s)^2] + \mathbb{E}[X(t)^2] - 2 \mathbb{E}[X(s) X(t)] \\
        &= 2 (\tilde{c}(0) - \tilde{c}(|s - t|)).
    \end{align}
    Suppose now that $X$ satisfies the ICA. Then, by the definition of the ICA, there is
    a function $f$ such that $X$ is $f$-H\"older and that $f$ satisfies the ICA. We also
    know that $2 (\tilde{c}(0) - \tilde{c}) \leq f$. Therefore, also $2 (\tilde{c}(0) -
    \tilde{c})$ satisfies the ICA. Conversely, suppose that $2 (\tilde{c}(0) -
    \tilde{c})$ satisfies the ICA. Then, $X$ satisfies the ICA using $2 (\tilde{c}(0) -
    \tilde{c})$ as $f$.
\end{proof}

\begin{proposition}\label{prop:lipschitz}
    Let $f$ be a nonnegative function on $[0, 1]$ that is non-decreasing in a
    neighborhood of $0$. Let $X$ be a $f$-H\"older continuous process. If $f$ satisfies
    the ICA, then there exists a non-decreasing continuous function $\phi$ on $[0, 1]$
    with $\phi(0) = 0$, which depends only on $f$, and a real-valued random variable $A$
    with bounded second moment such that for all $s, t \in [0, 1]$,
    \[
        |X(s) - X(t)| \leq A \phi(|s - t|) \,.
    \]
\end{proposition}
\begin{proof}
    The claim follows directly from Theorem 2.3 in \cite{hahn1977} with $r = 2$.
\end{proof}

\begin{corollary}\label{cor:lipschitz}
    Consider the setup from Proposition \ref{prop:lipschitz}. Let $X_1, \dots, X_n$ be
    a random sample with the same distribution as $X$. From the proposition we know
    there exists random variables $A_1, \dots, A_n$ and non-decreasing functions
    $\phi_1, \dots, \phi_n$ such that the claim holds. Define $\phi = \max_{i = 1}^n
    \phi_i$. Then, $\phi$ is non-decreasing, continuous, $\phi(0) = 0$, and for all $i =
    1, \dots, n$,
    \[
        |X_i(s) - X_i(t)| \leq A_i \phi(|s - t|) \,.
    \]
\end{corollary}
\begin{proof}
    The claim follows directly from Proposition \ref{prop:lipschitz}, the fact that
    $X_i$ has the same distribution as $X$, and the fact that the maximum of continuous
    non-decreasing functions is continuous and non-decreasing.
\end{proof}


\begin{theorem}\label{thm:fclt}
    Let $X_1, \dots, X_n$ be independent mean-zero processes following the same
    distribution as $X$. Suppose that $\mathbb{E}[X(t)^2] < \infty$ for all $t$, and
    that $X$ satisfies the ICA, and has $C^1[0, 1]$ sample paths almost surely. Then,
    \[
        \frac{1}{\sqrt{n}} \sum_{i = 1}^n X_i \overset{d}{\to} \mathcal{GP}(0, c)
        \quad \text{in} \, C^1[0, 1],
    \]
    with $c(s, t) = \mathbb{E}[X(s) X(t)]$.
\end{theorem}
\begin{proof}
    This follows directly by Theorem 2.5 in \cite{hahn1977}.
\end{proof}


\begin{theorem}\label{thm:uniform-convergence-kernel}
    Let $X_1, \dots, X_n$ be independent mean-zero processes following the same
    distribution as $X$. Suppose that $X$ satisfies the ICA and that $\sup_t X(t)^2 =
    \mathcal{O}_p(1)$. Let $\hat{c}(s, t) = \frac{1}{n} \sum_{i = 1}^n X_i(s)
    X_i(t)$ denote the sample covariance. Then,
    \begin{align}
        || \hat{c} - c ||_\infty &\overset{p}{\to} 0 \\
        || \partial_{1, 2}\hat{c} - \partial_{1, 2}c ||_\infty &\overset{p}{\to} 0
    \end{align}
    where $|| \cdot ||_\infty$ is the supremum norm. If we further assume that $\sup_t
    X(t)$ has finite second moment, convergence is almost surely.
\end{theorem}
\begin{proof}
    \emph{First claim.}
    Let $s, t, s', t' \in [0, 1]$. Let $A_1, \dots, A_n$ and $\phi$ be the random
    variables and function from Proposition \ref{prop:lipschitz} and Corollary
    \ref{cor:lipschitz}, respectively. We have
    \begin{align}
        |\frac{1}{n} &\sum_{i = 1}^n X_i(s) X_i(t) - \frac{1}{n} \sum_{i = 1}^n X_i(s') X_i(t')| \\
        &\leq |\frac{1}{n} \sum_{i = 1}^n (X_i(s) - X_i(s')) X_i(t) + (X_i(t) - X_i(t')) X_i(s')| \\
        &\leq \frac{1}{n} \sum_{i = 1}^n |X_i(s) - X_i(s')| |X_i(t)| + \frac{1}{n} \sum_{i = 1}^n |X_i(t) - X_i(t')| |X_i(s')| \\
        &=: I_1 + I_2 \,.
    \end{align}
    Now consider $I_1$. Using the Cauchy-Schwarz inequality, we have
    \begin{align}
        I_1 &\leq \left(\frac{1}{n} \sum_{i = 1}^n |X_i(s) - X_i(s')|^2\right)^{1/2}
        \left(\frac{1}{n} \sum_{i = 1}^n |X_i(t)|^2\right)^{1/2} \\
        &\leq \left(\frac{1}{n} \sum_{i = 1}^n |X_i(s) - X_i(s')|^2\right)^{1/2}
        \left(\frac{1}{n} \sum_{i = 1}^n \sup_t |X_i(t)|^2\right)^{1/2} \\
        &= \left(\frac{1}{n} \sum_{i = 1}^n \sup_t X_i(t)^2\right)^{1/2}
        \left(\frac{1}{n}\sum_{i = 1}^n |X_i(s) - X_i(s')|^2\right)^{1/2} \\
        &\leq \left(\frac{1}{n} \sum_{i = 1}^n \sup_t X_i(t)^2\right)^{1/2}
        \left(\frac{1}{n}\sum_{i = 1}^n A_i^2 \phi^2(|s - s'|)\right)^{1/2} \\
        &= \left(\frac{1}{n} \sum_{i = 1}^n \sup_t X_i(t)^2\right)^{1/2}
        \left(\frac{1}{n}\sum_{i = 1}^n A_i^2 \right)^{1/2} \phi(|s - s'|) \\
        &=: C_n \phi(|s - s'|) \,.
    \end{align}
    Doing the same calculations for $I_2$, and utilizing the $\ell_1-\ell_2$-norm
    inequality and that $\phi$ is non-decreasing, we get
    \begin{align}
        I_1 + I_2 &\leq C_n \left(\phi(|s - s'|) + \phi(|t - t'|)\right) \\
        &\leq C_n 2 \phi(|s - s'| + |t - t'|) \\
        &\leq C_n 2 \phi\left(2(|s - s'|^2 + |t - t'|^2)^{1/2}\right)
    \end{align}

    Now let $h(x) = 2 \phi(2 x)$, and notice that $h(x) \to 0$ as $x \to 0$, since
    $\phi$ is continuous and $\phi(0) = 0$. Thus, we have
    \[
        |\hat{c}(s, t) - \hat{c}(s', t')| \leq C_n h\left((|s - s'|^2 + |t -
        t'|^2)^{1/2}\right) \,.
    \]
    Let us start by showing the case of almost sure convergence. By Theorem 22.10 in
    \cite{davidson2021} we know that $\{\hat{c}_n \}$ is strongly stochastically
    equicontinuous if $\limsup_n C_n < \infty$ almost surely. We show this by showing
    that $C_n$ converges almost surely to a finite value. Recall that $A_1, \dots, A_n$
    are IID and have finite second moments. Thus by the strong law of large numbers and
    an application of the continuous mapping theorem we know that
    \[
        \left(\frac{1}{n} \sum_{i=1}^n A_i^2\right)^{1/2} \overset{a.s.}{\to}
        \mathbb{E}[A_1^2]^{1/2} < \infty \,.
    \]
    Similarly, as the $X_1, \dots, X_n$ are IID and we assumed that $\mathbb{E}[\sup_t
    X_1(t)^2] < \infty$ we know that
    \[
        \left(\frac{1}{n} \sum_{i=1}^n \sup_t X_i(t)^2\right)^{1/2} \overset{a.s.}{\to}
        \mathbb{E}[\sup_t X_1(t)^2]^{1/2} < \infty \,.
    \]
    And thus $C_n$ converges to a finite value almost surely, which gives the desired
    result. Following the discussion after Theorem 22.9 in \cite{davidson2021} we know
    that then also $\{\hat{c}_n - c\}$ is strongly stochastically equicontinuous. And
    finally, by Theorem 22.8 in \cite{davidson2021} we get
    \[
        ||\hat{c} - c||_\infty \overset{a.s.}{\to} 0 \,.
    \]
    Now let us consider the case of convergence in probability. In this case we know
    that $\sup_t X_i(t)^2 = \mathcal{O}_p(1)$ for all $i$, and thus
    \[
        \left(\frac{1}{n} \sum_i \sup_t X_i(t)^2\right)^{1/2} = \mathcal{O}_p(1) \,.
    \]
    Hence, $C_n = \mathcal{O}_p(1)$, and $\{\hat{c}_n\}$ is stochastically
    equicontinuous by Theorem 22.10 in \cite{davidson2021}. Thus, by Theorem 22.9 in
    \cite{davidson2021}
    \[
        ||\hat{c} - c||_\infty \overset{p}{\to} 0 \,.
    \]

    \emph{Second claim.}
    To be written.
\end{proof}


\begin{corollary}
    Consider the conditions of Theorem  \ref{thm:fclt}. Define $\tilde{X}_i(t)
    = X_i(t) / \sqrt{c(t, t)}$. Then
    \[
        \frac{1}{\sqrt{n}} \sum_{i = 1}^n \tilde{X}_i(t) \overset{d}{\to}
        \mathcal{GP}(0, \tilde{c}) \quad \text{in} \, C^1[0, 1],
    \]
    where $\tilde{c}$ is a correlation function satisfying $\tilde{c}(t, t) = 1$.
\end{corollary}
\begin{proof}
    Notice that the process $\tilde{X}_i$ is also mean-zero and satisfies the ICA. From
    Theorem \ref{thm:fclt} we know that
    \begin{align}
        \tilde{c}(s, t) &= \mathbb{E}[\tilde{X}_i(s)\tilde{X}_i(t)] \\
        &= \mathbb{E}\left[\frac{X_i(s)}{\sqrt{c(s, s)}} \frac{X_i(t)}{\sqrt{c(t, t)}}\right] \\
        &= \frac{c(s, t)}{\sqrt{c(s, s) c(t, t)}},
    \end{align}
    and thus $\tilde{c}$ is a correlation function.
\end{proof}
