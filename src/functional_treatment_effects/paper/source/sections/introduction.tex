\section{Introduction}


\subsection{Motivation and literature review}

\begin{center}
[... insert motivation and literature review here ...]
\end{center}

\subsection{Notation}

Let $\to_p$ denote convergence in probability, $\to_d$ convergence in distribution,
$\mathcal{N}(m, C)$ the (multivariate) normal distribution with mean $m \in
\mathbb{R}^p$ and covariance $C \in \mathbb{R}^{p \times p}$, $\mathcal{GP}(m, C)$ the
distribution of a Gaussian process with mean function $m:[0, 1] \to \mathbb{R}$ and
covariance function $C:[0, 1]^2 \to (0, \infty)$.


\subsection{Simultaneous Confidence Bands}

In the following we introduce a set of assumptions required by a functional estimator
that allow for the computation of simultaneous confidence bands. For this we utilize the
methods developed in \cite{liebl2021}. The following results require that the estimator
is asymptotically Gaussian, which we state in Assumption
\ref{ass:asymptotically_gaussian_estimator}.

\begin{assumption}
    \label{ass:asymptotically_gaussian_estimator}
    Suppose $\hat{\theta}$ is an estimator of the process $\theta = \{\theta(t) : t \in
    [0, 1]\}$ such that
    \begin{align*}
        \sqrt{n}(\hat{\theta} - \text{Bias}(\hat{\theta}) - \theta) \to_d
        \mathcal{GP}(0, C_\theta) \,,
    \end{align*}
    for some covariance function $C_\theta$.
\end{assumption}

Before we state the main theorem concerning the simultaneous confidence bands we define
the roughness parameter function and simultaneous confidence bands.

\begin{definition}
    \label{def:roughness_parameter_function}
    Consider the setup of Assumption \ref{ass:asymptotically_gaussian_estimator} and let
    $\tilde{C}_\theta(s, t) = {C_\theta(s, t)}/{\sqrt{C_\theta(s, s) C_\theta(t, t)}}$.
    Define the roughness parameter function\footnote{In \cite{liebl2021} the roughness
    parameter function is denoted by $\tau(t)$. We use $r(t)$ here because $\tau(t)$
    denotes our treatment effect function in section 2.?.}
    \[
        r(t) = (\partial_{1,2}\tilde{C}_\theta(t, t))^{-1/2} \,,
    \]
    which quantifies the extent of the local multiple testing problem.
\end{definition}

\begin{definition}
    \label{ass:simultaneous_confidence_band}
    Consider any confidence level $\alpha \in (0, 1)$. The two-dimensional random
    function $\text{SCB}_\alpha(t) = (\theta_\ell(t), \theta_u(t))$ is a correctly sized
    simultaneous confidence band for the parameter $\theta(t)$ if
    \[
        \mathbb{P}(\forall t \in [0, 1]: \theta(t) \in [\theta_\ell(t), \theta_u(t)])
        \geq 1 - \alpha \,.
    \]
\end{definition}

Using the roughness parameter function $r(t)$ the method of \cite{liebl2021} allows us
to compute a \textcolor{red}{fair} threshold function (critical value function)
$u_\alpha(t)$, depending on the confidence level $\alpha \in (0, 1)$. Theorem
\ref{thm:scb_for_estimator} states that using this threshold function leads to a valid
simultaneous confidence band.

\begin{theorem}
    \label{thm:scb_for_estimator}
    Let $u_\alpha(t)$ be a threshold function computed by Algorithm 1 from
    \cite{liebl2021} using the roughness parameter function $r(t)$. Let $\alpha \in (0,
    1)$ denote a confidence level and set $\text{SCB}_\alpha(t) = (\hat{\theta}_\ell(t),
    \hat{\theta}_u(t))$ for all $t \in [0, 1]$, with
    \begin{align*}
        \hat{\theta}_\ell(t) &= \hat{\theta}(t) - u_\alpha(t) \sqrt{C_\theta(t, t) /
        n}\\
        \hat{\theta}_u(t) &= \hat{\theta}(t) + u_\alpha(t) \sqrt{C_\theta(t, t) / n} \,.
    \end{align*}
    Then $\text{SCB}_\alpha$ constitutes a valid simultaneous confidence band.
\end{theorem}
\begin{proof}
    See Section 2.4.1, Example 1 in \cite{liebl2021}.
\end{proof}


Until now we assumed to know the roughness parameter function and the covariance
function. Under the following assumptions we can extend the result of Theorem
\ref{thm:scb_for_estimator} to the practically more relevant case where both $r(t)$ and
$C_\theta$ are estimated from data.

\begin{assumption}
    \label{ass:estimator_assumptions}
    Here I need to state a set of general assumption on the estimates that allow us to
    extend Theorem \ref{thm:scb_for_estimator}.
\end{assumption}

\begin{theorem}
    \label{thm:main_theorem}
    Consider the setup of Theorem \ref{thm:scb_for_estimator}, but using estimates
    $\hat{r}(t)$ and $\hat{C}_\theta$ for $r(t)$ and $C_\theta$, respectively. Under
    assumption \ref{ass:estimator_assumptions} the simultaneous confidence band
    $\text{SCB}_\alpha$ is still valid.
\end{theorem}
\begin{proof}
    To be written.
\end{proof}
